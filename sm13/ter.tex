%% Определение размера листа и шрифта
\documentclass{article}


\usepackage{header}
\title{Домашнее задание по курсу <<Математическая статистика>> от 10 марта}
\author{Хайдуров Руслан, 151}
\begin{document}
\maketitle
\section{Задание 1}
Воспользуемся методом моментов: $\E(X) = \overline{X}$. Поскольку случайная величина $X$ из бета-распределения, то ее первый и второй моменты $\E(X) = \frac{\alpha}{\alpha + \beta}$, $\D(X) = \frac{\alpha \beta}{(\alpha + \beta)(\alpha + \beta +1)}$. Для несмещенной оценки дисперсии воспользуемся такой оценкой:
$$
    S^2 = \cfrac{1}{n-1}\sum\limits_{k = 1}^{n}(X - \overline{X})^2
$$
Теперь сам метод моментов: мы заменяем параметры $\alpha$ и $\beta$ на искомые оценки $\alpha^*$ и $\beta^*$, и наша задача будет выразить их через $\overline{X}$ и $S^2$. Из первого выражения получим
$$
    1 - \overline{X} = \cfrac{\beta^*}{\alpha^* + \beta^*}
$$
Подставим это я выражение для $S^2$, и наша задача будет просто решить систему:
$$
	\begin{cases}
		S^2 = \cfrac{\overline{X}(1 - \overline{X})}{\alpha^* + \beta^* + 1}\\
		\overline{X} = \cfrac{\beta^*}{\alpha^* + \beta^*}
	\end{cases}
$$
\section{Задание 2}
Воспользуемся теоремой из курса теории вероятностей: для непрерывной функции $h(x)$ верно 
\end{document}